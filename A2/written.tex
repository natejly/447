%%%%%%%%%%%%%%%%%%%%%%%%%%%%%%%%%%%%
% This is an assignment template for
% CPSC 447/547. Last update: Feb 2022.
% How to use:
% Find line '\newcommand{\psetnumber}{}'
% set the value to an appropriate number.
% Find line '\newcommand{\name}{}'
% set the value to your full name.
% Find line '\newcommand{\psetnumber}{}'
% set the value to your NetID.
% Do not change other macro.
% You solution to each task should start with
% the heading command: \task{}
%%%%%%%%%%%%%%%%%%%%%%%%%%%%%%%%%%%%
\documentclass[12pt]{exam}
\usepackage{amsmath}
\usepackage{enumerate}
\usepackage[left=1in, right=1in, top=1in, bottom=1in]{geometry}
\newcommand\Q[1]{\question{\large{\textbf{#1}}}}
\newcommand{\task}[1]{\rule{\linewidth}{1pt}
\noindent{\bf Task~ #1}\vspace{-0.4em}\\
\rule{\linewidth}{1pt}\vspace{.10in}}
\newcommand{\answerbox}[1]{
\begin{framed}
\hspace{5.65in}
\vspace{#1}
\end{framed}}

% Assignment Number (Change this)
\newcommand{\psetnumber}{1}
% Name and ID (Change this)
\newcommand{\name}{Name}
\newcommand{\netid}{NetID}

\pagestyle{head}
% header
\headrule \header{\textbf{CPSC 447/547 Assignment \psetnumber}}{\name}{\textbf{Page
\thepage\ of \numpages}}
\pointsinmargin \printanswers
\setlength\answerlinelength{2in} \setlength\answerskip{0.3in}


\begin{document}
\addpoints
\begin{center}
%title
\textbf{\large{Assignment \psetnumber}}\\
% date
\vspace{0.1in}
\today\\
\vspace{0.1in}
 Name:  \name \\
 NetID:  \netid
\end{center}


\begin{center}
\textit{Collaborator(s): None} % Write your collaborator's name here, if any
\end{center}

\begin{questions}
\setcounter{question}{0} % set initial question counter, if needed
\Q{Fun with Quantum Computing}

% Task 1.1
 \task{1.1}
You solution here.

% Task 1.2
 \task{1.2}
You solution here.

% Task 1.3
 \task{1.3}
You solution here.





\end{questions}
\end{document}